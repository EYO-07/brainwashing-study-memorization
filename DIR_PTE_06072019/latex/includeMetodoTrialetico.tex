% includeMetodoTrialetico

%%%%%%%%%%%%%%%%%%%%%%%%%%%%%%%%%%%%%%%%%%%%%%%%%%%%%%%%%%%%%%%%%%%%%%%%%%%%%%%%%%%%%%%%%%%%%%%%%%%%%%%%%%%%
%%%%%%%%%%%%%%%%%%%%%%%%%%%%%%%%%%%%%%%%%%%%%%%%%%%%%%%%%%%%%%%%%%%%%%%%%%%%%%%%%%%%%%%%%%%%%%%%%%%%%%%%%%%%
%%%%%%%%%%%%%%%%%%%%%%%%%%%%%%%%%%%%%%%%%%%%%%%%%%%%%%%%%%%%%%%%%%%%%%%%%%%%%%%%%%%%%%%%%%%%%%%%%%%%%%%%%%%%
%%%%%%%%%%%%%%%%%%%%%%%%%%%%%%%%%%%%%%%%%%%%%%%%%%%%%%%%%%%%%%%%%%%%%%%%%%%%%%%%%%%%%%%%%%%%%%%%%%%%%%%%%%%%
%%%%%%%%%%%%%%%%%%%%%%%%%%%%%%%%%%%%%%%%%%%%%%%%%%%%%%%%%%%%%%%%%%%%%%%%%%%%%%%%%%%%%%%%%%%%%%%%%%%%%%%%%%%%

\chapter{Método Trialético de Estudo}

\hspace{\baselineskip}

O método trialético de estudo consiste na consciência de três etapas distintas no processo de aprendizagem de uma teoria. As três etapas são a memorização, a análise e a aplicação, sendo que tais etapas são feitas preferencialmente nesta ordem. No estudo tradicional temos duas etapas distintas, a primeira é a exposição do conteúdo feita por meio da leitura ou por meio de um orador, nesta etapa não se exclui o diálogo, a interação é permitida, a segunda etapa é a etapa de aplicação, em que trabalhos, exercícios e aplicações são propostos. A diferença entre o estudo tradicional e o estudo trialético é que a exposição do conteúdo é decomposto em duas etapas, a memorização e a análise, enquanto a etapa de aplicação permanece inalterada.

%%%%%%%%%%%%%%%%%%%%%%%%%%%%%%%%%%%%%%%%%%%%%%%%%%%%%%%%%%%%%%%%%%%%%%%%%%%%%%%%%%%%%%%%%%%%%%%%%%%%%%%%%%%%
%%%%%%%%%%%%%%%%%%%%%%%%%%%%%%%%%%%%%%%%%%%%%%%%%%%%%%%%%%%%%%%%%%%%%%%%%%%%%%%%%%%%%%%%%%%%%%%%%%%%%%%%%%%%
%%%%%%%%%%%%%%%%%%%%%%%%%%%%%%%%%%%%%%%%%%%%%%%%%%%%%%%%%%%%%%%%%%%%%%%%%%%%%%%%%%%%%%%%%%%%%%%%%%%%%%%%%%%%
%%%%%%%%%%%%%%%%%%%%%%%%%%%%%%%%%%%%%%%%%%%%%%%%%%%%%%%%%%%%%%%%%%%%%%%%%%%%%%%%%%%%%%%%%%%%%%%%%%%%%%%%%%%%
%%%%%%%%%%%%%%%%%%%%%%%%%%%%%%%%%%%%%%%%%%%%%%%%%%%%%%%%%%%%%%%%%%%%%%%%%%%%%%%%%%%%%%%%%%%%%%%%%%%%%%%%%%%%

\section{Motivação e Propósito}

\hspace{\baselineskip}

Este paradigma foi feito (ou refeito por mim, eu não sei, reflita sobre isso) no século 21, mas o problema que motiva a criação de novos métodos de estudo acompanha o homem estudioso desde algum tempo, talvez desde que o homem começou a querer documentar as informações da natureza por meio da escrita e por meio de símbolos. 

O estudioso já deve ter notado que o seu cérebro é limitado na capacidade de assimilar novas informações, e o seu desejo por entender é confrontado com a realidade de que seu cérebro não é o melhor método de armazenar conteúdos, seu cérebro erra, seu cérebro tem velocidade de assimilação limitada, seu cérebro tem velocidade de acesso limitada, seu cérebro não foi feito para lidar com excesso de informações. 

Para você leitor que assim como eu recebeu a dádiva de ter uma memorização fotográfica de quatro pixels ou menos, mas um desejo ousado por entender talvez toda a natureza (no sentido de incluir também a cultura humana parte da natureza) então este método é para você, não é um método para pessoas medíocres, mas também não é um método para prodígios (nada impede que seja usado, ora pois).\\

\textbf{Motivação [ Método Trialético de Estudo ]} Excesso de informação e limitação humana.\\

Temos duas soluções para este problema, aceitar a limitação humana e usar a estratégia de dividir para conquistar, cada um se especializa como conseguir o seu cérebro, o que pode levar a possível dificuldade de comunicação entre áreas muito distintas, mas ainda é uma estratégia realista e eficiente. Outra solução é tentar lidar com o excesso de informações enfrentando o problema com métodos mais eficientes de estudo.\\

\textbf{Propósito [ Método Trialético de Estudo ]} Assimilar e entender teorias com conteúdos difíceis de memorizar, seja por excesso de informações importantes, seja por similaridades entre as informações importantes que podem levar a frequência de erros e instabilidade, seja porque você quer saber com detalhes um conteúdo qualquer mesmo que isso te leve a uma crise existencial.\\

É importante destacar que não é exclusividade de nenhuma área mais do intelecto humano o excesso de informações, muito se engana aquele que busca na matemática uma teoria completamente racional, porque irá ser confrontado com a necessidade de memorizar teoremas, como se não bastasse também entende-los e aplica-los de forma correta, isso vale para física, para engenharia, para química, para biologia, para história, para filosofia. 

Também é importante ter a consciência de que vivemos um mundo competitivo, é necessário buscar se sobrepor em qualidade ao outro para estar em uma posição de destaque, estar nela sem ter devido conteúdo é arrogar para si um direito do qual não é digno, podemos nos questionar se existiria um sistema diferente, mas até mesmo para propor um novo sistema é necessário estudar com detalhes os outros sistemas, é necessário ter a argumentação necessária para poder criar um sistema confiável que possa substituir o sistema atual, tentar fazer sem estudo, sem esforço não é somente arrogância, é irresponsabilidade. Por hora viver fora da mediocridade é uma necessidade convergente.

%É seguro dizer que o desenvolvimento intelectual está intimamente ligado a escrita que foi um método mais eficiente de armazenar informações sobre o mundo, a eficiência que se traduz em estabilidade da informação (informações mais precisas) e a facilidade de acesso.

%%%%%%%%%%%%%%%%%%%%%%%%%%%%%%%%%%%%%%%%%%%%%%%%%%%%%%%%%%%%%%%%%%%%%%%%%%%%%%%%%%%%%%%%%%%%%%%%%%%%%%%%%%%%
%%%%%%%%%%%%%%%%%%%%%%%%%%%%%%%%%%%%%%%%%%%%%%%%%%%%%%%%%%%%%%%%%%%%%%%%%%%%%%%%%%%%%%%%%%%%%%%%%%%%%%%%%%%%
%%%%%%%%%%%%%%%%%%%%%%%%%%%%%%%%%%%%%%%%%%%%%%%%%%%%%%%%%%%%%%%%%%%%%%%%%%%%%%%%%%%%%%%%%%%%%%%%%%%%%%%%%%%%
%%%%%%%%%%%%%%%%%%%%%%%%%%%%%%%%%%%%%%%%%%%%%%%%%%%%%%%%%%%%%%%%%%%%%%%%%%%%%%%%%%%%%%%%%%%%%%%%%%%%%%%%%%%%
%%%%%%%%%%%%%%%%%%%%%%%%%%%%%%%%%%%%%%%%%%%%%%%%%%%%%%%%%%%%%%%%%%%%%%%%%%%%%%%%%%%%%%%%%%%%%%%%%%%%%%%%%%%%

\section{Filosofia e Projeções Dialéticas}

\hspace{\baselineskip}

\textbf{Questão [ Externalização Trialética ]} Por que o nome deste documento é Externalização Trialética? O que diabos vem a ser Externalização e o que diabos vem a ser Trialética?\\

Caso você ainda não tenha percebido, o método trialético não é justificado cientificamente, tão pouco é uma teoria lógica formal, a única justificação que tem é com base na filosofia, o que é fraco para os padrões de objetividade atuais, mas paciência, porque é o que tem para hoje.

Uma resposta rudimentar para a pergunta é que o documento é resultado da aplicação do método trialético, é uma consequência da aplicação do método trialético para uma determinada teoria (coleção de informações), é usando um jargão próprio uma externalização do método trialético.

\subsection{Dialética Socrática}

%%%%%%%%%%%%%%%%%%%%%%%%%%%%%%%%%%%%%%%%%%%%%%%%%%%%%%%%%%%%%%%%%%%%%%%%%%%%%%%%%%%%%%%%%%%%%%%%%%%%%%%%%%%%
% 
\hspace{\baselineskip}



\textbf{Questão [ Dialética ]} O que é dialética? O que é trialética?\\

\textbf{Paradigma [ Dialética Socrática Clássica ]} Sócrates foi um filósofo grego que criou um método para analisar problemas chamado de diálogo ou dialética que consiste de duas etapas tradicionalmente conhecidas como ironia e maiêutica. Existem alguns elementos que precisam estar presentes na prática da filosofia socrática, primeiro deve haver pelo menos duas pessoas, o inquiridor e o interlocutor, o primeiro tem como papel fazer questionamentos sobre determinado assunto próprio do interlocutor (que o interlocutor supostamente domina), Sócrates fazia o papel de inquiridor, o segundo é uma pessoa que acha que domina determinado assunto. O inquiridor deve fazer perguntas detalhadas o suficiente para que a segunda pessoa se declare ignorante a respeito do que ele sabia, este é o objetivo da ironia, desmascarar a ignorância, a segunda etapa da maiêutica é auxiliar de forma propositiva o interlocutor para que ele mesmo possa achar a resposta, mas a resposta deve vir do próprio interlocutor, Sócrates apenas auxilia a encontrar o caminho. A etapa da ironia é encapsulado na frase célebre atribuído a Sócrates: "Só sei que nada sei, e o fato de saber isso, me coloca em vantagem sobre aqueles que acham que sabem alguma coisa".\\

\textbf{Questão [ Arrogância x Ignorância ]} No sentido atribuído a Sócrates, qual a diferença entre ignorância e arrogância? \\

\textbf{Paradigma [ Dialética Socrática ]} Fazendo uma releitura da dialética Socrática, defino como Dialética Socrática o procedimento de enfrentar um problema através de dois processos, o primeiro processo é o processo de questionamento, que requer uma sensibilidade, uma capacidade de perceber inconsistências, de perceber insuficiências, de aumentar a resolução das perguntas para ser possível fazer questionamentos mais profundos, e o segundo processo é o processo de proposição, em que possíveis soluções são apresentadas para responder as questões, tais proposições não são definitivas, toda resposta tem uma pergunta inerente, a resposta é verdadeira ou não? Fazendo um paralelo temos que o questionamento seria a etapa de ironia, e a proposição seria a etapa de maiêutica, entretanto existem diferenças pontuais, primeiro não é necessário duas pessoas, segundo a divisão em duas etapas não implica que as duas etapas precisam estar separadas, mas sim que existem dois processos que podem ser chamados de forma arbitrária de acordo com a necessidade do problema.\\

\textbf{Paradigma [ Dividir para Conquistar ]} Também conhecido como dividir e governar, atribuído a história-política como uma estratégia de governo em dividir o poder em poderes menores subordinados para conseguir controlar um reino/império. Em ciência da computação é um paradigma de dividir um determinado problema em subproblemas menores e mais fáceis de resolver.\\

\textbf{Questão [ Dialética Socrática e Dividir para Conquistar ]} Qual a relação entre dividir para conquistar e a dialética socrática?\\

A dialética socrática também é um processo de decomposição de um problema em subproblemas, assim a aplicação do método socrático herda os mesmos benefícios do paradigma de Dividir para Conquistar, ele pode simplificar uma questão complicada em questões menores. Entretanto a dialética socrática não consiste em apenas simplificar, mas também em aumentar a resolução, ou seja, refinar os problemas encontrando mais problemas que não se tinha consciência de existirem. Dividir para Conquistar tem um propósito incutido, quero resolver o problema, quero encontrar a solução, enquanto a dialética socrática, pode ser usada para encontrar soluções ou para aumentar os problemas. Dialética Socrática se encaixa na academia enquanto que Dividir para Conquistar se encaixa na sociedade em geral. Do meu ponto de vista a Dialética Socrática estende o paradigma de Dividir para Conquistar.

Ter consciência do método socrático talvez seja o mais importante que você possa aprender neste capítulo, caso não tenha dificuldades de posse do método socrático você já pode fazer muita coisa, apenas usando o pensamento natural, a aprendizagem natural. Caso você não esteja satisfeito com seu desempenho então é interessante refinar o seus métodos, utilizar outros paradigmas que é o objetivo central deste capítulo.\\

\subsection{Projeções Dialéticas}

\hspace{\baselineskip}

%%%%%%%%%%%%%%%%%%%%%%%%%%%%%%%%%%%%%%%%%%%%%%%%%%%%%%%%%%%%%%%%%%%%%%%%%%%%%%%%%%%%%%%%%%%%%%%%%%%%%%%%%%%%
% 
\textbf{Questão [ Dialética Materialista e Projeções Dialéticas ]} O que tem a ver a dialética materialista com a dialética socrática?\\

Bem, este é um ponto interessante, porque a dialética materialista por si só foi capaz de gerar a forma mais terrível de governo com base no Ideário do Socialismo Marxista, e continua a dar muita dor de cabeça para a política atual. Mesmo com essa bagagem negativa, vejo que existe uma ideia interessante sob o ponto de vista filosófico, mas que é preciso aparar alguns aspectos para que não gere problemas para o futuro. 

Os socialistas devem estar contorcendo o nariz com o que disse acima, eu peço que não olhe o socialismo como exclusividade do Marxismo, por exemplo o catolicismo, a igreja católica sempre foi uma instituição socialista, sempre buscou ajudar os outros, a se solidarizar com os outros, o Marxismo sequestrou o socialismo (ou pelo menos as pessoas que são atraídas pelo socialismo) e agora os socialistas sofrem de uma síndrome de estocolmo, tem empatia pelo seu sequestrador, eu sugiro que se você ficou em qualquer nível incomodado com minhas palavras que pense em criar um socialismo do zero, não busque o que não deu certo, se liberte.

Voltando ao que interessa, a dialética materialista é como se fosse um modelo que analisa um determinado processo através da interação entre dois objetos, o objeto inicial chamado de tese, um segundo objeto chamado de antítese, e um terceiro elemento resultante que na verdade é a absorção da tese pela antítese ou da antítese pela tese, esse terceiro elemento é chamado de síntese. Podemos entender a dialética socrática através da dialética materialista, a tese seria uma afirmação, e a antítese seria uma afirmação contrária, na dialética materialista o conflito tem um papel importante, assim as duas afirmações contrárias gerariam um conflito, que é o questionamento, e como são opostos, um ou outro estaria certo, assim um ou outro seria absorvido ou eliminado para gerar a síntese que seria o ganhador do processo. A primeira falha da dialética materialista é sua mistificação, esse processo se tornou mistificado, como se fosse uma fórmula mágica, um segredo do universo, e isso fez com que o socialismo marxista se tornasse uma religião, a segunda falha é a ênfase no conflito, as interações duais não são sempre conflituosas, nem sempre há necessidade da destruição do outro, a terceira falha é que mesmo no conflito deve existir um propósito comum, um guia que unifique os processos, o objetivo não é a destruição, na dialética socrática é chegar em uma solução.

Olhando de forma geral o que faz a dialética materialista, ela representa um processo por meio de componentes, na matemática existe um processo de representação de entidades matemáticas através de uma sequência de outros objetos, por exemplo uma função pode ser representada por uma soma de tipos especiais de funções multiplicadas por constantes, esses tipos especiais são chamados de base e as constantes associadas a cada elemento da base são chamados de componentes ou projeções. Unificando as ideias, na dialética materialista você busca representar um processo por meio da base (tese, síntese, antítese), por que não podemos ter bases com números arbitrários de elementos?\\

\textbf{Paradigma [ Projeções Dialéticas ]} Uma base dialética é uma sequência de componentes (a,b,c,...) que são usadas para modelar algo, chamamos o processo de modelagem de projeção dialética, nela buscamos os componentes específicos de cada entrada da base. Por exemplo: Na ironia da dialética socrática sob a ótica da dialética materialista (base dialética materialista) o problema (questão) é visto como o conflito entre duas ideias contrárias, a primeira ideia seria correspondente a tese e a segunda ideia seria correspondente a antítese que geraria o conflito.\\

\textbf{[ Estrutura ]} Uma coleção de objetos que possuem significado entre si, ou seja, existem propriedades entre estes objetos que caracterizam a estrutura. Dizemos que uma estrutura é preservada entre duas coleções se cada uma delas possuem estas mesmas propriedades, ou seja, o que caracteriza uma estrutura é seu significado e não os seus componentes em si.\\

\textbf{[ Base Dialética ]} Uma coleção de componentes que formam uma estrutura.\\

\textbf{Base Dialética [ Dialética Existencial ]} A dialética existencial olha a realidade como uma coleção de coisas, que irei chamar de existências, cada existência possui dois componentes, um componente tangível chamado de forma externa e um componente intangível chamado de caráter interno. A acessibilidade é a propriedade que caracteriza essa estrutura, o caráter interno somente é acessível através da forma externa para nossa consciência (ou para um objeto distinto qualquer do universo).\\

A dialética existencial modela o universo como um conjunto discreto (enumerável, contável) de objetos que chamamos de existências. Por conta disso a dialética existencial é elementar, ele está lidando com o que é mais elementar na consciência humana que é a representação do universo através de sua delimitação em partes menores distinguíveis entre si. 

A representação natural de uma existência é um ponto, ou uma bola, ou uma bola com um ponto dentro representando uma forma externa e possivelmente algo que não enxergamos como seu centro que é o caráter interno.\\

\textbf{Paradigma [ Discretização ]} Este paradigma diz que precisamos discretizar a realidade em objetos finitos, distinguíveis para poder entender o universo, o contínuo é visto como uma coisa só, ou como uma coleção de coisas delimitadas por fronteiras finitas, não conseguimos trabalhar bem com o contínuo apenas usando a imaginação (por exemplo não conseguimos medir distâncias precisamente, não conseguimos medir velocidades precisamente, entretanto enxergamos a diferença, assim podemos dizer o que é maior do que é menor, o que é mais rápido do que é lento). Isso justifica a modelagem em bases, isso justifica a representação do universo em existências, isso também justifica a matemática para trabalhar com medidas.\\

\subsection{Trialética e Fundamento de Quatro Posições}

\hspace{\baselineskip}

\textbf{Base Dialética [ Trialética Existencial ]} A trialética existencial é uma extensão da dialética existencial, a trialética vê uma existência como um processo de interação, temos a recepção, um estimulo que chega a forma externa e alcança seu interior, e o interior responde com uma mudança na forma externa, essa mudança é chamada de resposta. Vamos chamar o terceiro elemento de processamento na falta de outro termo, assim temos a trialética existencial como (Recepção, Processamento, Resposta). A propriedade estrutural é que os padrões de recepção e resposta caracterizam a existência, que os padrões de recepção e resposta é uma forma de tangenciar o caráter interno da existência, e que o processamento é a parte mais próxima do caráter interno. Portanto os padrões de recepção e resposta caracterizam a existência. \\

Note que a trialética existencial pode ser vista pela dialética existencial, a (recepção, resposta) corresponderia a forma externa do processamento, o processamento seria uma entidade mais interna, não exatamente o caráter interno, mas algo que está comunicando diretamente com o caráter interno.\\

\textbf{Questão [ Existencial ]} Algo existe se não pode interagir? Qual a diferença entre algo que não existe e algo que não tem efeito nenhum sobre nosso universo? \\

\textbf{Base Dialética [ Trialética Humana ]} A trialética humana é uma projeção do ser humano como existência na trialética existencial, entretanto ela mesma é uma base dialética, vamos chamar (Emoção, Intelecto, Vontade) como a projeção da mente humana na (Recepção, Processamento, Resposta), é importante notar que estamos projetando a mente humana e não o corpo humano, a mente humana, a consciência humana ainda é interior ao nosso corpo. A emoção lida com a recepção de nossa consciência, o intelecto lida com o processamento, e a vontade lida com a resposta voluntária de nossa consciência que promove uma resposta (ou não) do corpo.\\

\textbf{Base Dialética [ Fundamento de Quatro Posições ]} O fundamento de quatro posições tem origem religiosa (encontrada no livro Princípio Divino, na religião conhecida como Igreja da Unificação ou Unificacionismo), mas assim como a dialética materialista ela será abordada de uma forma mais objetiva e menos metafísica. O fundamento de quatro posições busca corrigir a base dialética materialista adicionando um quarto elemento, este na religião é conhecida como Deus, mas que pode ser abordado mais objetivamente como a origem do propósito, ou propósito. Todo processo acontecendo no tempo e no espaço é uma interação entre dois elementos distribuídos no espaço, um elemento chamado de sujeito, mais ativo, possui mais iniciativa, e um elemento chamado de objeto, mais passivo e temos dois elementos associados a transformação temporal, o propósito e o resultado, ficamos então com (propósito, sujeito, objeto, resultado), entretanto a transformação temporal pode ser vista trialeticamente como (Origem, Divisão, União), este é a projeção temporal do fundamento de quatro posições, a origem corresponde ao propósito, a divisão corresponde a interação entre sujeito e objeto e a união é o resultado. O sujeito tem uma conexão mais íntima com o propósito enquanto o objeto tem conexão mais íntima com o resultado. Como o propósito é subjetivo o método de estudo será trialético, pois definir um propósito é particular do leitor, entretanto é interessante ter consciência de que é preciso de um propósito em suas ações, o ser humano deve agir de forma consciente para fazer jus a sua existência.\\

\textbf{[ Ação de Dar e Receber ]} A ação de dar e receber é um processo intrínseco da estrutura do fundamento de quatro posições, ela descreve o processo de interação entre o sujeito e o objeto, quando a interação se torna estável então dizemos que é formado uma nova existência, a existência resultante, ela pode ser de fato uma nova existência ou pode ser a internalização da interação entre o sujeito e objeto formando um novo centro, onde a própria interação se torna uma existência.\\

É interessante observar que as noções de tempo e espaço estão incutidas no fundamento de quatro posições, o espaço descrito como sujeito e objeto e o tempo descrito como o processo de origem, divisão e união. O caráter temporal apesar de ser um atributo da estrutura do fundamento de quatro posições, pode também ser atribuído a qualquer trialética, uma trialética tem uma característica temporal inerente. Quanto ao caráter espacial, ele pode ser visto como essencialmente dialético, dual, ou pode ser visto como uma coleção de elementos que podem estabelecer uma relação dialética. Essa característica do Fundamento de Quatro posições conter tanto dialética quanto trialética vem de um dos objetivos que justificam o propósito de sua criação, de que o fundamento de quatro posições seria assim como a dialética materialista, uma verdade universal, sua ambição é ser uma fórmula geral para as projeções dialéticas.


%%%%%%%%%%%%%%%%%%%%%%%%%%%%%%%%%%%%%%%%%%%%%%%%%%%%%%%%%%%%%%%%%%%%%%%%%%%%%%%%%%%%%%%%%%%%%%%%%%%%%%%%%%%%
%%%%%%%%%%%%%%%%%%%%%%%%%%%%%%%%%%%%%%%%%%%%%%%%%%%%%%%%%%%%%%%%%%%%%%%%%%%%%%%%%%%%%%%%%%%%%%%%%%%%%%%%%%%%
%%%%%%%%%%%%%%%%%%%%%%%%%%%%%%%%%%%%%%%%%%%%%%%%%%%%%%%%%%%%%%%%%%%%%%%%%%%%%%%%%%%%%%%%%%%%%%%%%%%%%%%%%%%%
%%%%%%%%%%%%%%%%%%%%%%%%%%%%%%%%%%%%%%%%%%%%%%%%%%%%%%%%%%%%%%%%%%%%%%%%%%%%%%%%%%%%%%%%%%%%%%%%%%%%%%%%%%%%
%%%%%%%%%%%%%%%%%%%%%%%%%%%%%%%%%%%%%%%%%%%%%%%%%%%%%%%%%%%%%%%%%%%%%%%%%%%%%%%%%%%%%%%%%%%%%%%%%%%%%%%%%%%%

\section{Digressão Filosófica sobre a Consciência Humana}

\hspace{\baselineskip}

\subsection{Memória}

\hspace{\baselineskip}

\textbf{Questão [ Memória e Consciência ]} No paradigma trialético de estudo queremos dar uma atenção especial a memorização, queremos resolver o problema do excesso de informações. É interessante, antes de afirmar qualquer, coisa entender como funciona a nossa memória, como armazenamos conteúdo, como acessamos conteúdo. Quais os problemas, o que pode inibir o armazenamento de conteúdo, o que pode inibir o acesso do conteúdo?\\

Vamos tentar resolver o problema com o método rudimentar da filosofia de projeções dialéticas. Rudimentar no sentido de que poderia ser feito com base na neurociência, mas que não pretendo faze-lo no momento. Mas antes vamos estabelecer uma notação.\\

\textbf{Notação [ Projeção Dialética ]} 

$$ \texttt{Objeto} \prec \texttt{Projeção} \bullet \texttt{Base Dialética} $$

$$ \texttt{Objeto} \prec \texttt{Componentes} \bullet \texttt{Base Dialética} $$

$$ \texttt{Objeto} \prec (a,b,c) \bullet (A,B,C) $$

Com esta notação queremos dizer que o Objeto está sendo visto como uma projeção sob a luz de uma base dialética.

$$ (a,b,c) \sim (f,g,h) $$ 

$$ (a,b,c) \bullet (f,g,h) $$

Representa a preservação de uma mesma estrutura entre as coleções (a,b,c) e (f,g,h).

$$ \succ \texttt{Objeto} \bullet \texttt{Base Dialética} ? $$

Significa a determinação do problema de encontrar os componentes/projeções do objeto sob a ótica da base dialética.

$$ A := \texttt{Algo mais Longo} $$

O simbolo $:=$ representa definição, no sentido usado acima representa a encapsulação de um conceito em uma representação mais simples.

$$ A \prec B \bullet C \bullet D $$

É importante notar que interno e externo é um conceito relativo, assim o terceiro elemento D se refere a quem está sendo usado como referência na comparação, a notação acima deve ser visto como A visto como B projeção sob a ótica de C base dialética relativo a D.  Outra forma de enxergar o conceito é interpretar C como um intermediário de comunicação entre B e D.

\hspace{\baselineskip}

De posse desta notação iremos realizar a análise dialética, a modelagem dialética, ou projeções dialéticas usando a representação dos conceitos simbolicamente, de forma que possamos nos libertar da linearidade natural do texto para algo com caráter mais espacial.

%%%%%%%%%%%%%%%%%%%%%%%%%%%%%%%%%%%%%%%%%%%%%%%%%%%%%%%%%%%%%%%%%%%%%%%%%%%%%%%%%%%%%%%%%%%%%%%%%%%%%%%%%%%%
% Projeção Dialética da Memória
\hspace{\baselineskip}

\textbf{Projeção Dialética [ Memória e Bases Existenciais ]} 

$$ \succ \texttt{Memória} \bullet \texttt{Trialética Existencial} ? $$
$$ \texttt{Memória} \prec \texttt{(Recepção, Armazenamento, Acesso)} \bullet \texttt{Trialética Existencial} $$
\hrulefill

Podemos enxergar a memória como um processo de recepção, armazenamento e acesso de conteúdo.\\

$$ \succ \texttt{(Recepção, Armazenamento, Acesso)} \bullet \texttt{Dialética Existencial} ? $$
$$ \texttt{(Armazenamento, \{Recepção, Acesso\})} \sim \texttt{(Caráter Interno, Forma Externa)} \sim \texttt{Dialética Existencial} $$
$$ \texttt{(Qualidade de Armazenamento, \{Qualidade de Recepção, Qualidade de Acesso\})} \sim \texttt{Dialética Existencial} $$
\hrulefill

A qualidade de armazenamento pode ser verificada pela qualidade da recepção e pela qualidade de acesso.\\

$$ Q1:= \texttt{Qualidade de Recepção} $$
$$ Q2:= \texttt{Qualidade de Armazenamento} $$
$$ Q3:= \texttt{Qualidade de Acesso} $$
$$  \texttt{Memória} \prec (\{Q1,Q2\} , Q3) \bullet \texttt{Dialética Existencial} \bullet \texttt{Consciência} $$
\hrulefill

Para a nossa consciência podemos enxergar a qualidade da recepção e a qualidade de armazenamento de conteúdo através da qualidade do seu acesso. Queremos ter uma boa qualidade de recepção, ter uma boa qualidade de armazenamento, entretanto somente temos consciência através da qualidade de seu acesso. 

%%%%%%%%%%%%%%%%%%%%%%%%%%%%%%%%%%%%%%%%%%%%%%%%%%%%%%%%%%%%%%%%%%%%%%%%%%%%%%%%%%%%%%%%%%%%%%%%%%%%%%%%%%%%
% 

\hspace{\baselineskip}

\subsection{Consciência}

\hspace{\baselineskip}

\textbf{Questão [ Consciência ]} Supostamente entendemos que a memorização se constitui de três processos (recepção, armazenamento, acesso), sendo mais rigoroso, temos uma ideia fraca do que seria, temos uma heurística para podermos trabalhar. Mas como realmente funciona estes três processos?

\begin{itemize}
	\item[ (i) ] \textbf{Questão [ Acesso ]} Que entidade está acessando o conteúdo na memória?
	\item[ (ii) ] \textbf{Questão [ Recepção e Armazenamento ]} Que entidade está controlando a recepção e o armazenamento?
\end{itemize}

Temos uma noção de que a mente não se constitui de algo abstrato, mas de algo concreto chamado cérebro e de algo abstrato que podemos chamar de consciência. O acesso estaria mais próximo da consciência, a entidade mais interior da existência humana, enquanto a recepção e o armazenamento estaria mais próximo do cérebro. Ou poderíamos afirmar que temos a consciência, uma existência virtual, abstrata que se comunica através de um mesmo processo de recepção e resposta com uma entidade concreta que armazena o conteúdo, o cérebro, as conexões entre os neurônios.

$$ \texttt{Consciência} \bullet \texttt{Cérebro} \bullet \texttt{Corpo} \bullet \texttt{Realidade} $$

Isto é uma organização topológica das entidade Consciência, Cérebro, Corpo e Realidade. Ela representa a topologia, a relação de proximidade entre as entidades. Para representações mais gerais podemos usar o grafo, que basicamente é uma coleção de entidades e uma coleção de relações entre as entidades.

\hspace{\baselineskip}

\textbf{Notação [ Interação ]}

$$ \texttt{Objeto A}  \left( \begin{array}{c}
\texttt{Recepção sob o Ponto de Vista de A} \\ \texttt{Ação sob o Ponto de Vista de A}
\end{array} \right) \bullet \texttt{Objeto B} $$

$$ \texttt{Objeto A}  \left( \begin{array}{c}
\texttt{Recepção sob o Ponto de Vista de A} \\ \texttt{Ação sob o Ponto de Vista de A}
\end{array} \right) \bullet \texttt{Base} \bullet \texttt{Objeto B} $$

A notação segue a regra da mão direita o de cima é para esquerda e o de baixo é para direita no conteúdo dos parênteses, ele representa a interação do objeto A com B sob o ponto de vista de A.

$$ \texttt{Objeto A} \bullet \left( \begin{array}{cc}
\texttt{Ação de B} \\ \texttt{Recepção de B}
\end{array} \right) \texttt{Objeto B}  $$

A notação acima também segue a regra da mão direita, note que a relação se inverte.

%%%%%%%%%%%%%%%%%%%%%%%%%%%%%%%%%%%%%%%%%%%%%%%%%%%%%%%%%%%%%%%%%%%%%%%%%%%%%%%%%%%%%%%%%%%%%%%%%%%%%%%%%%%%
% Análise dialética da Consciência
\hspace{\baselineskip}

\textbf{Análise Dialética [ Consciência ]}

$$ C := \texttt{Consciência}$$
$$ R := \texttt{Realidade} $$
$$ M := \texttt{Memória} $$
$$ C \left(  \begin{array}{cc}
\texttt{Percepção} \\ \texttt{Vontade}
\end{array} \right)  \bullet \left( \begin{array}{cc}
\texttt{Sentidos Captados} \\ \texttt{Alteração da Realidade}
\end{array} \right) R $$
$$ C \left(  \begin{array}{cc}
\texttt{Imaginação} \\ \texttt{Acesso:Informações}
\end{array} \right)  \bullet \left( \begin{array}{cc}
\texttt{Envio:Informações} \\ \texttt{Memorização}
\end{array} \right) M $$
\hrulefill

\textbf{Questão [ Intermediário ]} Existe uma pergunta topológica natural neste modelo, será que a memória atua conectando a realidade com nossa consciência como um intermediário exclusivo ou será que em nossa arquitetura cerebral temos duas vias de comunicação, uma via que se comunica diretamente com a realidade e uma via que se comunica com a memória e possibilita a imaginação?

%%%%%%%%%%%%%%%%%%%%%%%%%%%%%%%%%%%%%%%%%%%%%%%%%%%%%%%%%%%%%%%%%%%%%%%%%%%%%%%%%%%%%%%%%%%%%%%%%%%%%%%%%%%%

\subsection{Raciocínio, Evolução e Fundamento de Quatro Posições}

\hspace{\baselineskip}

\textbf{Questão [ Raciocínio e Memorização ]} Existe raciocínio sem memorização?\\

\textbf{Questão [ Substrato do Raciocínio ]} O raciocínio ocorre somente na imaginação ou ele ocorre também na realidade?\\

%%%%%%%%%%%%%%%%%%%%%%%%%%%%%%%%%%%%%%%%%%%%%%%%%%%%%%%%%%%%%%%%%%%%%%%%%%%%%%%%%%%%%%%%%%%%%%%%%%%%%%%%%%%%
% Evolução
\textbf{Base Dialética [ Evolução ]} Temos a evolução como uma estrutura de três componentes (Existências, Transformações, Sentido), a existência é o objeto que evolui, as transformações são as possíveis mudanças nestes objetos, e o sentido é a direção da mudança. A transformação conecta uma existência com outra e possui uma direção. A evolução é um processo temporal, ela precisa de um substrato espacial, mas essencialmente ela é temporal. Algo somente pode evoluir se tiver estabilidade, a verdade é que algo somente existe se tiver alguma estabilidade de forma que seja possível estabelecer propriedades que externalizem sua existência para um ser consciente.

$$ \texttt{(Informação, Raciocínio, Verdade)} \sim \texttt{Evolução} $$

%\textbf{Notação [ Existencial ]} (Caráter Interno/Natureza Interna)\{Forma Externa\} \\

\textbf{Análise Dialética [ Consciência, Cérebro, Memória, Realidade, Raciocínio, Imaginação ]}

$$ \mathcal{C} := \{ \texttt{Consciência, Cérebro, Memória, Realidade, Raciocínio, Imaginação} \} $$
$$  \succ \mathcal{C} \bullet \{ \texttt{Fundamento de Quatro Posições} \} ? $$
$$ FQP := \texttt{Fundamento de Quatro Posições} $$
$$ \mathcal{C} \prec \texttt{(Natureza Interna, Corpo, Realidade, Memória)} \bullet FQP $$
\hrulefill

A natureza do ser humano impele o cérebro (componente do corpo) a interagir com a realidade através do corpo, o resultado da interação estável de dar e receber com a realidade cria um novo centro, uma nova existência chamada de memória, ela é externalizada pelas conexões que os neurônios fazem para associar diferentes locais da memória.

$$ \texttt{(Natureza Interna, Memória, Realidade, Consciência da Realidade)} \sim FQP $$

\hrulefill

Uma vez existindo a memória pode interagir com as outras existências. A natureza interna estimula a relação entre a memória e a realidade, o estabelecimento de uma relação estável de dar e receber cria uma nova entidade, a consciência da realidade, a percepção da realidade. Supostamente todos os animais possuem esse grau de consciência. Podemos chamar a consciência da realidade de percepção.

$$ \texttt{(Natureza Interna, Corpo, Memória, Imaginação)} \sim FQP $$

\hrulefill 

A relação estável entre a memória e o cérebro cria outra forma de consciência, a consciência de nossas experiências armazenadas na memória, que chamo de imaginação. Podemos agora definir a consciência de duas maneiras, a consciência é a resultante da interação dual entre (Imaginação, Percepção) que seria uma existência dialética ou podemos definir a consciência com uma existência trialética, em que as três entidades interagem dualmente entre si formando um equilíbrio entre três entidades o equilíbrio entre (Corpo, Realidade, Memória).

Com a consciência temos dois substratos naturais para evolução da informação (raciocínio), o substrato da realidade, ou seja, o raciocínio é feito/acompanhado através da transformação dos objetos da realidade, e o substrato da imaginação em que a memória permite simular a realidade e o raciocínio se dá na memória.

Agora o suprassumo da consciência:

$$ \texttt{(Consciência, Raciocínio da Imaginação, Raciocínio na Realidade, Linguagem)} \sim FQP $$

\hrulefill

Ora temos agora um caminho natural para o desenvolvimento da linguagem, ela seria um intermediário resultante do raciocínio feito na imaginação e o raciocínio feito na realidade. Essa é uma leitura da linguagem de forma que seja utilizada para efetuar raciocínio lógico, mas a linguagem é bem mais que isso, pois está associado a interação entre seres humanos.\\

É importante perceber que a ação de dar e receber requer tempo, pois requer um processo de interação até chegar a estabilidade, assim a consciência requer tempo até que fique estável, mas existe um processo que é o mais importante para o método trialético de estudo, a informação resgatada da memória requer tempo para ficar estável no substrato criado da consciência da imaginação, é importante associar o tempo como uma dificuldade inerente, associar a estabilidade como um processo de disputa entre ideias, o entendimento disso permitirá criar um método eficiente de estudo.\\

\textbf{Questão [ Existência Trialética ou Dialética ]} A consciência humana é uma existência trialética (estabilidade de três elementos) ou é uma existência dialética (estabilidade de dois elementos)?\\

\textbf{Análise Dialética [ Informação e Verdade ]}

$$ \texttt{(Consciência, Informação(Consciência), Fenômeno(Realidade), Verdade)} \sim FQP $$

\hrulefill

A verdade é um atributo da informação, a verdade somente faz sentido com algo externo que possa comparar informações contraditórias, este ocupado pela realidade. A interação da consciência com a realidade permite a consciência do conceito de verdade.\\

\textbf{Questão [ Sonho e Realidade ]} Qual a diferença entre as percepções do Sonho e as percepções da Realidade?\\

A realidade tem regras mais rígidas em relação a verdade, em relação a direção dos fenômenos, o sonho não necessariamente segue as mesmas regras da realidade, algumas vezes contradiz, algumas vezes não tem sentido algum.\\

\hspace{\baselineskip}

\subsection{Escolha e Livre Arbítrio}

\hspace{\baselineskip}

\textbf{Questão [ Residência da Consciência ]} Onde reside a consciência? A consciência está mais próxima da realidade, está mais próxima da memória, está mais próxima do corpo?\\

Ambas as formas de consciência (Percepção, Imaginação) tem uma entidade em comum, o raciocínio, o raciocínio parece ser elementar para consciência, podemos dizer que se a consciência está mais próxima de algum lugar, deve ser do raciocínio. É interessante notar que o raciocínio tem uma direção definida e que não percebemos raciocínios simultâneos, sempre vemos o raciocínio como único, até quando imaginamos processos simultâneos imaginamos eles como se fosse um raciocínio único. Essa unicidade será assumida como característico da consciência. Quem estimula a unicidade da consciência quando desperto é a realidade, quem estimula a consciência quando dormindo seria a memória da realidade, a perda de unicidade do raciocínio provavelmente (e isso é uma das muitas especulações feitas até agora) faz com que percamos a consciência, então temos a seguinte hipótese, quando dormimos perdemos a unicidade, entre o estado desperto e o sono temos uma breve etapa em que imaginamos coisas sem sentido e dormimos, essas coisas sem sentido são consequência da perda da unicidade, antes de acordar, durante o sono profundo temos o sonho que seria a recuperação da unicidade feita pelo cérebro, quando ficamos completamente conscientes durante o sono geralmente acordamos, ou seja, recuperamos a unicidade do raciocínio na consciência.

Essa hipótese de unicidade do raciocínio é fundamental para o método trialético, pois ela configura uma disputa de ideias, ela permite entender o que é confusão, o que é desconcentração, o que é perda e dificuldade de raciocínio, pois tudo nada mais é do que a perda de unicidade do raciocínio. Portanto a perda de unicidade do raciocínio pode ser uma dificuldade inerente de qualquer arquitetura de consciência, até mesmo se criamos uma inteligência artificial, talvez essa disputa entre os raciocínios para definir o raciocínio dominante seja uma externalização do livre arbítrio, característico da consciência e que perturba e incentiva qualquer filósofo a estudar a mente humana.\\

\textbf{Paradigma [ Escolha ]} A escolha é uma direção única e definida para realizar uma ação. É intrínseco da consciência a escolha, a liberdade, o livre arbítrio. A escolha, a liberdade talvez seja o real centro da consciência, escolhemos como raciocinar na imaginação e escolhemos como agir (que seria o raciocínio na realidade).\\

\textbf{Questão [ Raciocínio e Ação ]} Dizemos que precisamos de uma direção única, mas o raciocínio e a ação são feitas simultaneamente ou uma vez fazemos uma e uma vez fazemos outra? Existe uma disputa entre o raciocínio e a ação?\\

Não tenho segurança em responder a esta pergunta, o que posso dizer que quando nos concentramos em agir estamos de certa maneira raciocinando, e quando nos concentramos em imaginar é como se estivéssemos agindo em um simulador, parece haver uma disjunção mas parece também que estamos percebendo estes fenômenos ao mesmo tempo.\\

\textbf{Análise Dialética [ Projeção da Consciência em Raciocínio ]}

$$ \texttt{(Consciência, Direção do Raciocínio)} \sim \texttt{Dialética Existencial} $$

\hrulefill

A consciência é externalizada pela direção de raciocínio, ela está mais próxima da consciência.\\

\textbf{Questão [ Perturbadores ]} O que pode estimular e alterar a direção do raciocínio?\\

Temos os estímulos da realidade percebida através dos sentidos, temos a sensibilidade interna de nosso corpo correspondente a emoção, temos a memória que influencia na imaginação, ou podemos chamar de intelecto, temos a vontade que está associado com a realidade. Eis que surge naturalmente a trialética humana (Emoção, Intelecto, Vontade) como os pertubadores da consciência, eles direcionam os raciocínios dominantes na consciência e os raciocínios não dominantes (releitura da subconsciência). Emoção está associada as sensações do corpo, aos desejos sensoriais, intelecto está associado a memória, a busca por entender e modelar a realidade, a vontade está na escolha voluntária, nas ações. Emoção e Vontade estão mais próximos da realidade enquanto o intelecto está mais próximo da memória. Podemos dizer que a consciência é o equilíbrio trialético existencial da emoção, do intelecto e da vontade externalizado sob a forma do livre arbítrio. Também podemos dizer que é um equilíbrio dialético existencial entre a realidade (Emoção, Vontade) e a memória (Intelecto).




%%%%%%%%%%%%%%%%%%%%%%%%%%%%%%%%%%%%%%%%%%%%%%%%%%%%%%%%%%%%%%%%%%%%%%%%%%%%%%%%%%%%%%%%%%%%%%%%%%%%%%%%%%%%
%%%%%%%%%%%%%%%%%%%%%%%%%%%%%%%%%%%%%%%%%%%%%%%%%%%%%%%%%%%%%%%%%%%%%%%%%%%%%%%%%%%%%%%%%%%%%%%%%%%%%%%%%%%%
%%%%%%%%%%%%%%%%%%%%%%%%%%%%%%%%%%%%%%%%%%%%%%%%%%%%%%%%%%%%%%%%%%%%%%%%%%%%%%%%%%%%%%%%%%%%%%%%%%%%%%%%%%%%
%%%%%%%%%%%%%%%%%%%%%%%%%%%%%%%%%%%%%%%%%%%%%%%%%%%%%%%%%%%%%%%%%%%%%%%%%%%%%%%%%%%%%%%%%%%%%%%%%%%%%%%%%%%%
%%%%%%%%%%%%%%%%%%%%%%%%%%%%%%%%%%%%%%%%%%%%%%%%%%%%%%%%%%%%%%%%%%%%%%%%%%%%%%%%%%%%%%%%%%%%%%%%%%%%%%%%%%%%\\

\section{Método Trialético de Estudo}

\hspace{\baselineskip}

\textbf{Questão [ Teoria ]} O que é uma teoria?\\

%Uma pergunta desse tipo para um filósofo dialético é o mesmo que induzir ao filósofo projetar o conceito "teoria" em alguma base dialética. Vamos nos contentar em dizer que uma teoria é uma trinca (Definições, Informações, Raciocínios), uma teoria está externalizada sob a forma de uma linguagem assim definições e informações devem estar externalizadas em forma de linguagem. Uma definição é composta de um nome e de uma coleção de propriedades, uma definição é sempre verdadeira, pois é uma criação, uma informação associa definições, elas podem ser verdadeiras ou falsas, propriedades são informações, uma definição é uma informação que associa um nome a propriedades, raciocínio conecta informações, um raciocínio pode ser visto também como informações e assim pode ser verdadeiro ou falso.\\

\textbf{[ Interpretação de Teoria como Grafo ]} Um grafo é uma coleção de nodos e uma coleção relações entre os nodos, vamos considerar uma teoria como uma coleção de nodos de informações e que as relações são raciocínios ou paradigmas que conectam as informações. É importante perceber que o raciocínio também é uma informação, assim representar uma teoria por meio de grafos é arbitrário.\\

\textbf{[ Grafo Existencial de Teoria ]} Aqui temos uma teoria como uma seleção de informações importantes, a função de uma teoria sob este ponto de vista é enfrentar problemas com tais informações. As relações que podem existir são, relação de cada teoria com um tópico ou categoria (por exemplo o nome da teoria, ou alguma subcategoria), o nome da categoria se relaciona com as informações presentes na teoria, esses nomes são estimuladores da consciência para lembrar das informações da teoria. O outro tipo de relação é a relação que existe entre as informações das teorias, suponha que força de existência é sua capacidade de influenciar o raciocínio principal de sua consciência (podemos chamar de força de consciência), tal relação é externalizada em quanto uma informação contribui para existência da outra informação. Existem tipos importantes de Grafos Existenciais para a memorização, a primeira é a linear, a segunda é a uniforme, e a terceira é a topologia de praticidade (ou de aplicação).

\subsection{Método Tradicional de Estudo}

\hspace{\baselineskip}

\textbf{[ Método Tradicional de Estudo ]} Consiste em duas etapas bem definidas, a primeira é a exposição do conteúdo e a segunda etapa é a exercitação do conteúdo através da resolução de exercícios e problemas. A exposição pode ser feita por um orador ou pela leitura de um livro. \\

\textbf{[ Método Trialético de Estudo ]} A primeira etapa é decomposta em duas, a memorização e a análise, enquanto a terceira etapa é a de exercitação, similar ao método tradicional.\\

\textbf{Notação [ Pertubação ]} 

$$ A \overset{ativa}{\rightarrow} B $$

Significa que A contribui positivamente para existência e/ou estabilidade de B, B tende a existir com A.

$$ A \overset{inibe}{\rightarrow} B $$

Significa que A contribui negativamente para existência e/ou estabilidade de B, assim B tende a não existir com A.

\hspace{\baselineskip}

\textbf{Análise Dialética [ Excesso de Informações ]}

$$  (\texttt{Excesso de Informações} \overset{inibe}{\rightarrow} \texttt{Memorização} ) \overset{inibe}{\rightarrow} \texttt{Consciência das Informações}  $$

$$ ( (\overset{inibe}{\rightarrow} \texttt{Consciência das Informações}) \overset{inibe}{\rightarrow} \texttt{Análise} ) \overset{inibe}{\rightarrow} \texttt{Aplicação}  $$

\hrulefill

O excesso de informações inibe o aprendizado, pois cria um efeito em cadeia inibitório que resulta em uma capacidade de aplicar o conhecimento ineficiente, como a externalização do aprendizado é a capacidade de aplicação então podemos dizer que uma capacidade deficiente de aplicação é seguramente um indicador de aprendizado deficiente.\\

\textbf{Paradigma [ do Aprendizado Natural ]} O aprendizado natural se dá durante a exposição, o processo de memorização e análise é feito ao mesmo tempo, não há uma consulta prévia do conteúdo aprendido, ou seja, o conteúdo da exposição é novidade.\\

\textbf{Questão [ Problema do Aprendizado Natural ]} O processo de memorização e análise é feito ao mesmo tempo de maneira livre durante a etapa de exposição do método tradicional (aprendizado natural). Quais problemas podem surgir quando estamos em um contexto de excesso de informação?\\

No aprendizado natural a compreensão do conteúdo é externalizada na capacidade de interação entre o expositor e o aprendiz, ou seja, o aprendiz tem capacidade de formular questões, dúvidas e proposições (Quando não há um expositor, mais o conteúdo é exposto em um livro a externalização se dá na capacidade de aplicar a dialética socrática). Quando há um excesso de informações há inibição da capacidade analítica, ou seja, o aprendiz não consegue formular questões e proposições porque não tem consciência do conteúdo, pois a taxa de informações é maior que a capacidade de se fixar na memória e assim poder ser usado pela consciência (raciocínio).\\

Para resolver este problema usamos o paradigma de dividir para conquistar, separamos o aprendizado natural em memorização e análise. Vamos definir que memorização seria artificial no sentido de que é penoso e trabalhoso, e a análise e aplicação seria natural no sentido que é prazeroso e mais fácil.\\

\textbf{Paradigma [ Grafos e Decomposição do Aprendizado Natural ]} Supondo a representação de uma teoria em grafos e lembrando que categorizar em nodo ou em relação uma informação é arbitrário, podemos usar um critério de escolha dos nodos e das relações, a definição dos nodos são as informações que serão memorizadas na etapa de memorização, a definição das relações são as informações secundárias resultantes da análise do conteúdo memorizado (dialética socrática).\\

\textbf{[ Grafo Existencial ]} Vamos agora abordar outra forma de ver uma teoria, suponha que os nodos sejam informações, pela digressão dialética da consciência nas seções anteriores uma informação para estar na consciência deve interagir e tornar-se estável, isso é resultado da relação entre a consciência e a memória. Também foi dito que a consciência é caracterizada por uma unicidade da direção do raciocínio, e que as informações tem a capacidade de perturbar a direção do raciocínio. Ora, temos que o raciocínio liga duas informações e que as informações tem a capacidade de pertubar a direção do raciocínio, uma conclusão possível é que uma informação contribui para existência de outra informação e isso é externalizado como raciocínio. Portanto estas relações de contribuições de existências entre as informações são as relações do grafo existencial. Chamamos de grafo existencial um grafo em que os nodos são existências e que as relações são as relações de contribuições (estabilidade e força de existência) entre as existências.\\

\textbf{[ Topologia Linear ]} É um grafo existencial em que as informações formam uma cadeia Linear de contribuições de existências.\\

\textbf{[ Topologia Uniforme ]} É um grafo existencial completo em que as informações estimula-se entre si com mesmo peso (mesma força de contribuição).\\

\textbf{[ Topologia de Aplicação ]} É um grafo existencial que tem pesos de relações diferentes para diferentes conjuntos de informações, tais relações são mais fortes quando uma informação e outra é estimulada ao mesmo tempo durante uma determinada aplicação. \\

\textbf{[ Topologia de Praticidade ]} É um grafo existencial que as relações existenciais são pela praticidade durante a aplicação dos conteúdos. É uma forma de topologia de aplicação particular para resolver problemas rapidamente.\\

\subsection{Topologia Linear, Topologia Uniforme e Topologia de Aplicação}

\hspace{\baselineskip}

\textbf{Questão [ Topologias de Existências ]} Supondo que uma teoria seja representada por um grafo existencial, qual topologia é ideal para armazenar uma teoria?\\

Uma das respostas mais naturais seria a topologia de aplicação ou a topologia de praticidade, pois o propósito de uma teoria é se aplicado na realidade e queremos uma aplicação eficiente. Seria bom se a topologia de aplicação fosse tangível, a verdade é que não sabemos bem quando pode ser necessário um determinado padrão de raciocínio, pois não conhecemos todos os possíveis problemas da realidade, a topologia de praticidade pode ser herdada pela execução de exercícios. Suponha que o objetivo seja a capacidade de aplicar o conteúdo independente da eficiência, ora se não sabemos o que encontrar então o melhor a fazer é preservar um padrão uniforme, assim teríamos todas as possibilidades e poderíamos em tese enfrentar todos os problemas possíveis, a dificuldade inerente é que o esforço para memorização da topologia uniforme é maior que os outros, pois o grafo é completo, também que a contribuição uniforme pode gerar o efeito de confusão, ou seja, não há uma direção do raciocínio predominante e portanto há inibição da consciência externalizada pelo raciocínio. A topologia linear é a última topologia que queremos, primeiro pelo acesso, como o a estrutura é linear o acesso a seu conteúdo não é uniforme e o seu custo é proporcional a quantidade de informações (leia sobre memória ram em ciência da computação, leia também sobre estruturas de dados, listas encadeadas e arrays), como estamos lidando com o problema do excesso de informações isso é horrível, na prática a estrutura linear é como se não tivéssemos aprendido nada.\\

\textbf{Questão [ Topologia Linear ]} Qual topologia é preservada na leitura?\\

Um livro lido de ponta a ponta preserva uma topologia existencial linear (que é horrível), entretanto durante o sono essa topologia pode se alterar (durante o sono outras conexões são formadas), assim a leitura linear não necessariamente é ruim se você tiver uma boa noite de sono. Se você for um músico, sabe que a topologia linear é interessante no aprendizado de uma música, pois uma música é essencialmente linear (mas o aprendizado é diferente, no aprendizado de uma música muito difícil é necessário decompor a música em partes e depois juntar as peças).\\

\textbf{Questão [ Dispersão Temporal e Topologia Linear ]} Qual topologia é preservada se o aprendizado se dá durante longos períodos de tempo?\\

Sabemos que o tempo é essencialmente linear, então a topologia natural do tempo é a topologia linear, um conteúdo disperso em longo período de tempo irá preservar uma topologia linear, a real questão é se o sono pode reverter essa situação ou simplesmente precisamos de outra abordagem, mais artificial para modificar a topologia existencial. Um estudante de graduação deve estar ciente deste problema, pois se seu curso tiver muito conteúdo você chegará em períodos avançados da graduação com uma topologia linear e o mundo exigirá de você uma topologia uniforme ou de aplicação. \\

\textbf{Questão [ Tempo e Espaço ]} Vemos que o tempo tem caráter linear, em contraste, qual a topologia é natural para o espaço?\\

A consciência espacial requer a consciência de simultaneidade, poderíamos dizer que seria mais uniforme que o tempo, mas não seria totalmente uniforme porque nosso olho tem foco, não é um sensor multidirecional. Isto é importante, o espaço permite criar conexões mais uniformes, por causa da sua natureza simultânea, assim uma boa ferramenta para modificar e uniformizar a topologia do conhecimento é projetar o conhecimento em uma linguagem e coloca-lo no espaço (em um papel ou na tela do computador), isso permite que você junte caminhos de forma artificial e fortaleça com o uso da realidade e do sentido de visão.\\

\subsection{Esquecimento e Desconcentração}

%%%%%%%%%%%%%%%%%%%%%%%%%%%%%%%%%%%%%%%%%%%%%%%%%%%%%%%%%%%%%%%%%%%%%%%%%%%%%%%%%%%%%%%%%%%%%%%%%%%%%%%%%%%%
% 
\hspace{\baselineskip}

A análise do esquecimento é importante para criação de um método de estudo, pois não queremos um armazenamento instável de conteúdo, queremos ter a certeza de que a memorização e o entendimento podem ser acessados quando necessário e que são estáveis e precisos, algo que é difícil de obter com o cérebro humano.\\

\textbf{Questão [ Memória ]} Como funciona o acesso da memória pela consciência? Como funciona a memorização?\\

\textbf{Análise Dialética [ Memória ]}

$$ \texttt{Memória} \prec \texttt{(Conexões, Neurônios)} \bullet \texttt{Dialética Existencial} \bullet \texttt{Cérebro} $$

\hrulefill

Perguntar sobre como funciona a memorização é equivalente (externalizado) em como nosso cérebro forma conexões. Sabemos que memorizamos algo se podemos desencadear a informação completa através do estimulo de partes da informação. Sabemos que memorizamos um processo se podemos reviver o processo através do estado inicial do processo. Estas duas afirmações não estão aí por acaso, uma está associado ao espaço e outro está associado ao tempo. Portanto a externalização de uma informação temporal ou espacial se dá pela capacidade estímulos de parte da informação desencadear a informação completa.

O esquecimento por outro lado é quando mesmo que se estimule partes da informação o desencadeamento ou não acontece ou não é forte o suficiente para perturbar a consciência.

$$ C := \texttt{Consciencia} $$
$$ M := \texttt{Conceito de Memorização} $$
$$ E := \texttt{Conceito de Esquecimento} $$
$$ M.A \times M.I := \texttt{Conceito de Memória Ativa e Inativa} $$
$$ (C,M,E,M.A. \times M.I.) \sim FQP $$

\hrulefill

Existe uma parte da memória que está acessível de forma próxima a consciência enquanto outra parte da memória está inativa, distante da consciência. A parte ativa da memória "reconhece" as partes que compõe uma informação da memória, então quando estas partes (não necessariamente bem definidas) são estimuladas por alguma outra informação na consciência então a outra informação na memória ativa é acessada pela consciência.\\

\textbf{Questão [ Esquecimento ]} Quais as formas de aparente inativação de memórias (esquecimento)?\\

Por hipótese diria quando há uma informação mais forte, muito similar mas de direção contrária (diz coisas diferentes), uma informação compete com outra informação e uma delas é mais forte inibindo a outra. Podemos também ter uma informação muito distante, que requer uma sequência de ativações até chegar ao destino. Podemos ter uma informação temporal muito longa, que requer um acesso linear sequencial e por isso é sujeito a frequentes desvios e pertubações de outras ideias da mente.\\

\textbf{Análise Dialética [ Esquecimento ]}

$$ \texttt{Esquecimento} \prec (\texttt{Interação Destrutiva}, \texttt{Acesso Sequencial}) \bullet (\texttt{Espaço}, \texttt{Tempo}) $$

\hrulefill

É interessante fazer um paralelo com a ondulatória, em geral ondas diferentes mas similares (fases próximas) tem maior efeito destrutivo do que ondas menos similares (fases distantes). Quando temos informações similares, por conta da unicidade do raciocínio, elas irão competir pelos estímulos da consciência, por conta disso elas podem levar ao esquecimento, uma da outra. O acesso sequencial pode levar ao esquecimento se a distância entre o início e a informação final for muito longa, isso irá requerer tempo e a consciência estará sujeita a estímulos que desviem o caminho para lembrança.

\hspace{\baselineskip}

\textbf{[ Acesso da Memória por Similaridade ]} É o acesso da memória pelas partes de uma informação, ela é imediata desencadeada pela similaridade entre as informações\\

\textbf{[ Acesso da Memória por Temporalidade ]} É o acesso da memória por meio de uma sequência de eventos ordenados no tempo, o acesso é linear e consome tempo, não é imediato e por isso está sujeito a perturbações.\\

\textbf{Questão [ Sono e Temporalidade Discretizada ]} Existe algum mecanismo durante o sono que otimize os fenômenos temporais?\\

Se comparamos sonho com o estado desperto, vemos que o sonho é mais irregular, o acesso da memória parece ser mais susceptível a pertubações, enquanto o estado desperto parece ser mais regular, talvez por que o estímulo da realidade é regular, e essa regularidade serve como uma referência para o cérebro manter a unicidade do raciocínio. O que sugere que a constância da realidade pode ser uma fonte de concentração para nossa memória. Isso é importante, pois justifica o raciocínio simbólico efetuado pela matemática, justifica a escrita e o desenho utilizado em outras áreas. Todas essas ferramentas estão na realidade e parecem ajudar a manter a concentração durante um raciocínio complicado.\\


\subsection{Definição do Paradigma Trialético}

%%%%%%%%%%%%%%%%%%%%%%%%%%%%%%%%%%%%%%%%%%%%%%%%%%%%%%%%%%%%%%%%%%%%%%%%%%%%%%%%%%%%%%%%%%%%%%%%%%%%%%%%%%%%
% 
\hspace{\baselineskip}

\textbf{Paradigma [ Método Trialético de Estudo ]} Consiste em três etapas distintas de estudo, a primeira etapa que exige esforço maior é a da memorização, o objetivo desta etapa é criar uma acessibilidade mais próxima possível topologia uniforme. A segunda etapa consiste na análise que essencialmente se dá pela leitura e pela dialética socrática, a topologia associada é chamada de topologia dedutiva, pois entrelaça as informações em relações de dedução, implicação. A terceira etapa é a etapa de aplicação do conhecimento, o objetivo é alcançar a topologia existencial prática, em que os caminhos mais fáceis são fortalecidos. O método trialético de estudo ainda não está completamente definido, a seguir será detalhado com mais profundidade os paradigmas do método.

$$ \texttt{M.T.E.} \prec ( \texttt{Memorização}, \texttt{Análise} , \texttt{Aplicação} ) \bullet \texttt{Trialética}$$

\hspace{\baselineskip}

Quero que o método conserve a estrutura da trialética humana, na estrutura trialética humana a emoção busca beleza, harmonia, equilíbrio, o intelecto busca a verdade, a organização e a vontade busca a praticidade, a aplicabilidade. Quero que a memorização fique mais palatável a consciência atribuindo características da emoção, a memorização será como uma etapa artística, em que a organização, o equilíbrio, a harmonia e a beleza se tornem instrumentos que facilitem a memória.

$$ (\texttt{Memorização}, \texttt{Análise}, \texttt{Aplicação}) \sim (\texttt{Emoção}, \texttt{Intelecto}, \texttt{Vontade}) $$

\hspace{\baselineskip}

%%%%%%%%%%%%%%%%%%%%%%%%%%%%%%%%%%%%%%%%%%%%%%%%%%%%%%%%%%%%%%%%%%%%%%%%%%%%%%%%%%%%%%%%%%%%%%%%%%%%%%%%%%%%
% 

\textbf{Paradigma [ Ferramental ]} Uma teoria pode ser representada por um conjunto de pares (Nome, Informação), o nome deve estar na linguagem verbal (suponho que a linguagem verbal ou escrita está mais próximo da consciência humana, primeiro porque a fala é acessível, segundo porque aprendemos a linguagem mais cedo do que a representação abstrata da matemática), quanto a informação, ela pode estar em qualquer linguagem, desde que a informação esteja completa, de preferência o uso de abstrações, representações em fórmulas ou descrições mais curtas e simples. A representação de uma teoria em um conjunto de pares (Nome, Informação) é chamado de ferramental, com analogia clara a ferramenta. O conjunto dos nomes de uma teoria (podendo estar organizado em categorias de estimuladores) é chamado de inventário, o conjunto das informações é chamado de externalização do ferramental, ou externalização do inventário.

$$ \texttt{Teoria} \prec \{ \dots ( \texttt{Nome} , \texttt{Informação} ) \dots \} \bullet \texttt{Ferramental} \bullet \texttt{Dialética Existencial} $$

\hspace{\baselineskip}

\textbf{Paradigma [ Inventário ]} A memorização eficiente é externalizada para consciência humana pela sua acessibilidade, o método de inventário é uma forma de verificar a acessibilidade das informações, ele consiste em enumerar o máximo possível de conteúdo de um tópico. O tópico é o agente estimulador da consciência e a enumeração externaliza o acesso provocado pelo estimulador. O método do inventário pode ser feito sem externalizar o ferramental, neste procedimento não é necessário verificar se realmente o conteúdo foi memorizado, o que é avaliado aqui é uma certa acessibilidade com relação aos estimuladores, por isso digo que o inventário externalizada virtualmente a acessibilidade. \\

\textbf{Paradigma [ Externalização de Inventário ]} Depois de feito a externalização da acessibilidade de uma teoria por seu inventário, pode ser interessante ver se as informações são acessíveis corretamente, usando o inventário como estimulador o ferramental é externalizado para verificar o acesso concreto do conteúdo.\\

\textbf{[ Taxa de Acesso ]} A taxa de acesso mede a eficiência do acesso, primeiro se estabelece um número que representa o ideal de informações acessadas (eu utilizo 100, pois considero que uma pessoa que é capaz de enumerar 100 informações de um tópico, mesmo que não domine o tópico, aparenta domina-lo), tentamos escrever ou falar o máximo de conteúdos distintos do tópico, contamos e dividimos por esse número ideal. A taxa de acesso irá externalizar a acessibilidade. Podemos usar uma versão alternativa da taxa de acesso estabelecendo um tempo limite (utilizo 2 horas) e considerando a taxa de acesso como enumeração dividido pelo tempo. Com esta variável será possível acompanhar a evolução do processo de memorização. A primeira versão da variável chamo de Resolução Espacial da Memorização, a segunda versão chamo de Eficiência Temporal da Memorização. \\

\textbf{[ Taxa de Acesso Virtual ]} Mede o acesso virtual feito pelo inventário. A taxa de acesso virtual é interessante na véspera de prova em que você não tem muito tempo para verificar todas as informações. \\

\textbf{[ Taxa de Acesso Concreto ]} Mede o acesso concreto feito na externalização do inventário. O acesso concreto é fundamental para garantir o aprendizado, mas é preciso ser realista, algumas vezes você não tem tempo para fazer tudo.\\

\textbf{Paradigma [ Acesso Randômico x Acesso Sequencial ]} Queremos preservar uma topologia uniforme das informações durante o processo de memorização, para isso é interessante que o acesso seja o mais caótico possível, randômico se você acreditar que o cérebro é realmente randômico e o mundo tem processos aleatórios. Portanto durante a externalização do acesso pelo paradigma de inventário tente não preservar ordem alguma, e claro, não preserve a ordem linear.\\

\textbf{Problema [ Estabilização da Taxa de Acesso ]} Pode ocorrer que a taxa de acesso pare de evoluir e se estabilize em um patamar inferior ao desejado, este é o problema da estabilização da taxa de acesso. Para resolver este problema podemos refinar o estimulador usando o paradigma de estimuladores.\\

\textbf{Paradigma [ Estimuladores em Árvore ]} Durante a externalização da memória pelo inventário, inicialmente temos um estimulador da memória, que é o tópico principal, o assunto ou o nome da teoria, caso haja uma estabilização da taxa de acesso, podemos utilizar o paradigma de dividir para conquistar e criar outros estimuladores, tais estimuladores podem ser subcategorias da teoria. Para conseguir um acesso de 100 conteúdos podemos dividir para conquistar e criar 10 subcategorias, cada uma responsável pelo acesso de 10, na soma conseguimos acessar todos os 100 com menor esforço, pois temos 1-10 de cada sub-estimulador mais 1-10 do tópico principal e das subcategorias. Tanto a escrita dos estimuladores quanto a escrita do inventário pode ser feita no papel, no computador ou em qualquer recurso disponível da realidade, pois isso irá aumentar a concentração herdando a estabilidade da realidade, de preferência escreva o inventário e os estimuladores em ambientes distintos.\\

\textbf{[ Externalização Trialética ]} Depender da memória para acessar o conteúdo pode ser instável, porque não está preso a realidade e está sujeito a inibição pelo excesso de informação. Por isso é interessante que o conteúdo a ser memorizado esteja organizado e externalizado em algo concreto (este algo concreto é este documento, a Externalização Trialética). O papel da externalização trialética é ser uma fonte confiável de consulta para poder memorizar. Como o objetivo é memorizar, deve ser sucinto e condensado como um formulário.\\

Com isso a etapa da memorização se subdivide em três novas etapas, a etapa de construção do documento externalização trialética, a etapa de estudo para memorização e a etapa de checagem de pela escrita de inventário e dos estimuladores.
$$ \texttt{Memorização Trialética de Teoria} \prec (\texttt{Seleção}, \texttt{Disgestão}, \texttt{Invocação}) \bullet \texttt{Base Trialética} $$

Vamos agora definir a etapa interior, que agora está sendo chamada de digestão de conteúdo. A motivação do nome vem do processo de dividir para conquistar, temos uma teoria extensa, dividimos em partes, estudamos de forma randômica seu conteúdo para estabelecer uma conexão uniforme.

Da maneira como quero definir esta etapa, ela será a mais artificial de todas, pois o cérebro humano não é um bom gerador aleatório, e o conteúdo escrito é inerentemente sequencial, para que o acesso seja aleatório podemos utilizar softwares de memorização (exemplo ANKI, de memorização por repetição espaçada) ou construir pequenas rotinas para sugerir de maneira randômica o acesso a determinada informação, seja por repetição espaçada como no ANKI ou por algum algoritmo diferente de memorização (por exemplo podemos numerar as informações e usar um gerador aleatório no python para números inteiros).

Sugiro fortemente que se você tem alguma noção de programação e tenha tempo, crie seu próprio algoritmo, pois você poderá corrigir eventuais desvios de propósito. \\

\textbf{Paradigma [ Invocação ]} É o paradigma de ferramental e de árvore de estimuladores. Invocação Interna é apenas para o inventário e Invocação Externa é para o ferramental concreto. \\

\textbf{Paradigma [ Digestão ]} Utilizar um gerador aleatório (ou qualquer algoritmo) para estimular o acesso do conteúdo (o estimulador pode ser o nome da informação, pode ser uma parte da informação). Deve ser construído este gerador com base nas informações escritas na seleção (Externalização Trialética), por isso o documento deve ter alguma forma de organização para facilitar a consulta. O nome da implementação é chamado de digestor. \\

\textbf{Paradigma [ Otimização da Taxa de Acesso pelo Digestor ]} O digestor pode funcionar como auxiliar para aumentar a taxa de acesso. Podemos usar o paradigma de estimuladores em árvore, mas o digestor é o que permite menos subcategorias e um acesso mais direto ao conteúdo.\\

\textbf{Paradigma [ Seleção ]} O conteúdo deve ser otimizado, trabalhado esteticamente para que fique confortável e mais fácil de memorizar durante esta etapa, deve selecionar o conteúdo de forma que seja de fácil consulta para a etapa de digestão.\\

Para resumir o método temos:\\

\textbf{Projeção Dialética [ Método de Estudo Trialético ]}

$$ \texttt{Estudo} \prec \texttt{(Memorização, Análise, Aplicação)} $$
$$ \texttt{Memorização} \prec \texttt{(Seleção, Digestão, Invocação)} $$
$$ \texttt{Análise} \prec \texttt{(Leitura, Dialética Socrática)} $$
$$ \texttt{Aplicação} \prec \texttt{(Fazer Exercícios, Criar Problemas)} $$
$$ \{ \texttt{Análise} \sim \texttt{Aplicação} \} \prec \texttt{Análise de Exemplos}$$

As etapas de análise e aplicação não serão detalhadas, pois são mais naturais contidos no método tradicional. Entretanto tanto a análise quanto a aplicação são externalizados no formato do texto. Por exemplo nas etapas de análise externalizamos o estudo escrevendo as demonstrações das informações como demonstrações dialética socráticas (é omitido as respostas, apenas as perguntas e construções auxiliares), na etapa da aplicação selecionamos exercícios que achamos importantes ou interessantes.


%%%%%%%%%%%%%%%%%%%%%%%%%%%%%%%%%%%%%%%%%%%%%%%%%%%%%%%%%%%%%%%%%%%%%%%%%%%%%%%%%%%%%%%%%%%%%%%%%%%%%%%%%%%%
%%%%%%%%%%%%%%%%%%%%%%%%%%%%%%%%%%%%%%%%%%%%%%%%%%%%%%%%%%%%%%%%%%%%%%%%%%%%%%%%%%%%%%%%%%%%%%%%%%%%%%%%%%%%\\
%%%%%%%%%%%%%%%%%%%%%%%%%%%%%%%%%%%%%%%%%%%%%%%%%%%%%%%%%%%%%%%%%%%%%%%%%%%%%%%%%%%%%%%%%%%%%%%%%%%%%%%%%%%%
%%%%%%%%%%%%%%%%%%%%%%%%%%%%%%%%%%%%%%%%%%%%%%%%%%%%%%%%%%%%%%%%%%%%%%%%%%%%%%%%%%%%%%%%%%%%%%%%%%%%%%%%%%%%
%%%%%%%%%%%%%%%%%%%%%%%%%%%%%%%%%%%%%%%%%%%%%%%%%%%%%%%%%%%%%%%%%%%%%%%%%%%%%%%%%%%%%%%%%%%%%%%%%%%%%%%%%%%%

\subsection{Estruturação da Externalização Trialética}

\hspace{\baselineskip}

Até aqui discutimos a argumentação filosófica dos paradigmas que formalizam este tipo de documento, agora vamos estabelecer uma série de notações e formalizações para a escrita do texto do tipo externalização trialética.

\hspace{\baselineskip}

\textbf{[ Definição ]} A definição é a primeiro elemento da organização de uma teoria, ela é sempre verdadeira, ela representa uma criação, um modelo para uma existência, geralmente composto de uma representação, um nome e um conjunto de propriedades. Como criação, ela é sempre verdadeira. No texto tudo que estiver em colchetes representa o nome de uma definição, e este é a definição de definição.\\

\textbf{[ Forma Externa ]} Uma forma externa está associado a externalização na consciência, a forma externa é uma reformulação de alguma entidade em outra mais fácil de memorizar, mais abstrata, ou mais organizada. \\

\textbf{[ Informações ]} Se temos uma teoria lógica axiomática, temos as informações assumidas como verdadeira de axiomas, as informações demonstradas pela lógica são teoremas (de maior importância), proposições (de importância menor) e Lema (utilizado apenas pontualmente em algum problema, ou algo muito simples de demonstrar), vamos simplificar e assumir que todas as informações são teoremas (o que implica em uma topologia existencial uniforme das informações). Paradigmas são informações não demonstradas formalmente, mas justificadas de alguma maneira, ou apenas consideradas verdade.

\hspace{\baselineskip}

Sempre que desejarmos estabelecer uma informação, usamos \textbf{Teorema [ Nome da Informação ]} com o teorema podendo ser substituído por paradigma, lema, proposição, informação etc. No caso da definição não escrevemos definição, apenas usamos \textbf{[ Nome da Definição ]}. Temos então o elemento de informação da externalização trialética como \textbf{Tipo de Elemento [ Nome do Elemento ]}, perceba o título do documento.

\hspace{\baselineskip}

\textbf{[ Notação ]} É uma definição simples, apenas diz como uma outra definição deve ser representada simbolicamente.

\hspace{\baselineskip}

\textbf{[ Estrutura ]} É uma lista da forma $(a,b,c)_{nome1}^{nome2}(A,B,C)$ que representa uma estrutura, um conjunto de definições que possuem uma relação entre si. Como convenção estabeleço que $(a,b,c)$ são elementos e $(A,B,C)$ são conjunto de elementos. O que caracteriza uma estrutura é sua relação interna.\\

\textbf{Paradigma [ Estrutural ]} Existem definições que sempre andam juntas, quando isso ocorre é interessante fazer uma exposição de uma vez utilizando a o elemento textual estrutura. \\

\textbf{Estrutura [ Teoremas ]} Um teorema pode ser dividido internamente como um conjunto de informações assumidas como verdade (Chamado de Hipóteses)  e um conjunto de informações que são deduzidas (Chamado de Kernel), eles formam a estrutura de um teorema $(\texttt{Hipótese}, \texttt{Kernel})_{\texttt{Teorema}}$, todo teorema pode ser dividido em hipótese e kernel, mas alguns teoremas podem ser representados apenas pelo kernel.\\

\textbf{Notação [ Teoremas ]} Neste texto um teorema é representado pela forma:

$$\Phi(\texttt{...Hipóteses...}) : \texttt{...Kernel...}$$

\textbf{[ Hipótese ]} Algumas vezes é interessante organizar hipóteses vista com frequência em diferentes teoremas, assim temos um tipo de classe de definição chamado hipótese.\\

\textbf{[ Kernel ]} Da mesma forma que as hipóteses, podemos ter kernels iguais em diferentes teoremas.\\

\textbf{[ Teorema ]} É uma informação que possui um nome, um corpo constituído de hipóteses, kernel e uma demonstração lógica (nestes tipos de documento irei optar pela dialética demonstrativa para representar uma demonstração, a dialética demonstrativa não demonstra, mas expõe as construções e as perguntas socráticas necessárias para resolver o teorema). Outros nomes para equivalentes de teorema são proposições e lemas, o teorema é considerado mais importante que as proposições, e as proposições são mais importantes que lemas. \\

\textbf{[ Exercício ]} São exercícios, informações que podem ser deduzidas dos teoremas, das propriedades de uma definição ou diretamente dos axiomas, é arbitrário definir o que é um exercício do que é um teorema/paradigma, mas um dos critérios objetivos possíveis é sua complexidade e aplicabilidade.\\

\textbf{[ Externalização ]} São exercícios que são aplicação direta de algum teorema/paradigma ou são deduzidos diretamente de propriedades de uma definição qualquer.\\

\textbf{[ Dialética Demonstrativa ]} É a representação de uma demonstração por meio de construções e perguntas, geralmente associado a um teorema. A ideia da dialética demonstrativa é expor o caminho da demonstração e deixar como exercício a escrita da demonstração propriamente dita.  \\

\textbf{[ Demonstração ]} É a demonstração de um teorema, a exposição passo a passo dos raciocínios dedutivos para chegar a conclusão de um teorema, mostrando sua validade.\\

\textbf{[ Externalização Trialética ]} É o tipo de documento que usa o formalismo definido nesta seção. O seu título também obedece ao formalismo: \textbf{Externalização Trialética [ Nome da Teoria ]}.  \\

\textbf{[ Motivação ]} Algumas vezes é interessante expor a motivação de uso de uma definição ou de um teorema.\\

\textbf{Paradigma [ Índice e Árvore Estimuladora ]} O índice e sua estrutura pode e deve ser a externalização de uma árvore de estimuladores, quanto mais organizado o conteúdo, mas fácil será para a etapa de digestão efetuada por geradores aleatórios e algoritmos de memorização.\\

\textbf{Questão [ Redundância ]} Pode ser que uma informação possa estar em uma categoria, quanto estar em outra, é vantajoso escolher uma ou é mais vantajoso colocar nas duas? \\

Deixo isso como arbitrário, algumas vezes se você quiser manter a teoria como uma teoria axiomática (que é estritamente linear quando externalizada em um documento) pode ser interessante que não inclua em muitas subcategorias uma mesma informação, algumas vezes se o objetivo é memorização apenas, ou que a teoria não seja uma teoria axiomática então é mais interessante incluir uma informação em diferentes categorias relacionadas, pois preserva um padrão de acessibilidade desejável.\\

O resultado de um estudo sob o método trialético bem sucedido é um documento do tipo externalização trialética, chamamos de externalização trialética, pois é a externalização da memorização (construção de fórmulas, estruturas, hipóteses, informações organizadas e categorizadas no índice), é a externalização da análise (nas demonstrações, nas motivações, nos paradigmas) e por fim é a externalização da aplicação (nos exercícios, nos exercícios de aplicação e em possíveis exercícios criados).

%%%%%%%%%%%%%%%%%%%%%%%%%%%%%%%%%%%%%%%%%%%%%%%%%%%%%%%%%%%%%%%%%%%%%%%%%%%%%%%%%%%%%%%%%%%%%%%%%%%%%%%%%%%%
%%%%%%%%%%%%%%%%%%%%%%%%%%%%%%%%%%%%%%%%%%%%%%%%%%%%%%%%%%%%%%%%%%%%%%%%%%%%%%%%%%%%%%%%%%%%%%%%%%%%%%%%%%%%\\
%%%%%%%%%%%%%%%%%%%%%%%%%%%%%%%%%%%%%%%%%%%%%%%%%%%%%%%%%%%%%%%%%%%%%%%%%%%%%%%%%%%%%%%%%%%%%%%%%%%%%%%%%%%%
%%%%%%%%%%%%%%%%%%%%%%%%%%%%%%%%%%%%%%%%%%%%%%%%%%%%%%%%%%%%%%%%%%%%%%%%%%%%%%%%%%%%%%%%%%%%%%%%%%%%%%%%%%%%
%%%%%%%%%%%%%%%%%%%%%%%%%%%%%%%%%%%%%%%%%%%%%%%%%%%%%%%%%%%%%%%%%%%%%%%%%%%%%%%%%%%%%%%%%%%%%%%%%%%%%%%%%%%%

\section{Depressão e Burnout}

\hspace{\baselineskip}

Este tópico é extremamente importante, a depressão poderá lhe acompanhar ao longo de sua vida, seja porque a sociedade fornece estímulos negativos constantes, seja porque sua vida não corresponde a suas expectativas (este método é um exemplo, o método trialético surgiu para lidar com o desejo de saber mais e com a limitação humana no aprendizado tradicional), seja por ignorância externalizada sob a forma de má alimentação, má cuidado com a saúde, seja por uma rotina de vida irregular, seja porque está cercado de pessoas ruins. Acho que já enumerei vários fatores que podem contribuir para um estado depressivo, mas o que seria depressão? \\

\textbf{Notação [ Relação ]} $ A \sim B $ se existe uma relação entre $A$ e $B$\\

Vou agora fazer uma série de associações com a palavra depressão e outros conceitos, esse método é uma forma de brainstorm (caso não saiba o que é brainstorm, pesquise!). Brainstorm é basicamente criar um conjunto de estimuladores na esperança de que a mente organize os raciocínios associados.\\

\textbf{Análise Dialética [ Depressão ]}

$$ \texttt{Depressão} \sim \texttt{Consciência} $$
$$ \texttt{Depressão} \sim \texttt{Emoção} $$
$$ \texttt{Depressão} \sim \texttt{Vontade} $$
$$ \texttt{(Depressão, Ausência de Vontade)} \sim \texttt{Dialética Existencial} \sim \texttt{Vontade} $$
$$ \texttt{(Depressão, Sentimentos Ruins)} \sim \texttt{Dialética Existencial} \sim \texttt{Emoção} $$

\hrulefill

Aparentemente há uma clara externalização da depressão por meio da vontade e da emoção, mas e quanto ao intelecto? O intelecto pode sofrer do mal da ignorância, do mal julgamento, da soberba, mas será que há depressão intelectual? Talvez a depressão intelectual esteja associado ao medo, ao chamado bloqueio mental, mas a ausência de vontade e as sensações ruins são as externalizações da depressão mais desgastantes, pois como não estão no intelecto, aparentemente não fazer sentido. Vamos considerar então primeiro a depressão externalizada através da ausência de vontade e das sensações ruins.\\

\textbf{Questão [ Externalização da Depressão ]} Existe relação entre sensações ruins (emoção) e ausência de vontade? \\

Essa pergunta foi de certa maneira respondida, sensações ruins podem levar a ausência de vontade, entretanto se as sensações ruins forem exclusivas do mundo externo então o ser humano seria incapaz de resolver problemas, o problema maior é quando as sensações ruins estão no interior, por causas desconhecidas.\\

\textbf{Questão [ Ações sem Sentido ]} A depressão é externalizada na vontade apenas pela ausência de vontade?\\

Podemos ter vontades ruins, por exemplo: suicídio, homicídio, automutilação, degradação do próprio corpo, isolamento. Estes são ações que externalizam a depressão, portanto uma definição mais precisa da externalização da depressão é a ausência de vontades consideradas boas pela própria consciência.\\

\textbf{Questão [ Estudo e Depressão ]} Vamos agora particularizar o problema, dentro do contexto de estudo, o que pode causar depressão?\\

O que pode causar ausência de vontade e o que pode causar sensações ruins durante o estudo? Sabemos que o esforço sem resultado pode inibir a vontade, se o estudo for ineficiente e trabalhoso então irá inibir a vontade. Você precisa sentir que seu esforço está valendo a pena. Também sabemos que o esforço excessivo sem descanso pode esgotar o corpo, isso pode levar a sensações ruins, portanto é essencial tempos de descanso, tempos de lazer e distração. Existe um problema na sociedade moderna, por exemplo a ciência incremental, ciência incremental é a ausência de grandes descobertas, a ciência se desenvolve apenas com pequenas contribuições, entretanto o esforço para uma pequena contribuição é muito grande. Neste caso é preciso de uma mudança de perspectiva, a contribuição incremental de cada indivíduo da sociedade pode ser pequena se olhado individualmente, mas se olhado como um todo pode ser relevante, talvez o seu problema seja o desejo por trabalhar sozinho, talvez seja porque você não confie nas pessoas. \\

\textbf{Questão [ Asceticismo e Esforço ]} O excesso de esforço sem resultado pode causar depressão, mas uma vida de prazer pode levar também a depressão? A ausência de resultados, a ausência de conquistas pode levar a depressão?\\

Se uma pessoa gosta de sorvete e come todos os dias, todo tempo sorvete, do mesmo tipo, o do tipo preferido, é provável que ele deixe de gostar. A monotonicidade, a ausência de novidades pode provocar ausência de desejo, o saciamento do desejo também pode inibir o desejo. Mesmo uma vida rica e diversa irá se tornar monótona por que o cérebro irá atribuir similaridade ao comportamento, se a vida se estabilizar em algum padrão ela se tornará monótona, e ausência de padrões é algo muito difícil de acontecer, a realidade tem padrões, a realidade é limitada em novidades (ou a nossa realidade próxima, pois o universo é extenso e inacessível). 
Agora vem o pulo do gato, tenho como hipótese que exista um equilíbrio ótimo entre o Asceticismo e o Esforço para controlar a depressão. A pergunta que fica é onde se situa o equilíbrio, qual a proporção de lazer e qual a proporção de esforço é o ideal. Segundo pulo do gato, para mim não existe proporção ideal, ela é passível de mudança e deve ser controlado por meio de alguns sinais externos, se você se sente cansado, se você acha que está se esforçando muito então aumente o tempo de lazer, se você sente que você não sente mais prazer com as atividades que você gostava, reduza o tempo, reduza a frequência e aumente o esforço em sua vida. É preciso lembrar que você pode ganhar seu dia apenas com um segundo, então minimizar o lazer não é uma coisa ruim para o ser humano, algumas vezes pode ser bom, mas a ausência completa pode levar a depressão. Talvez a pessoa que consiga maximizar o esforço e minimizar o prazer seja uma pessoa mais feliz, pois irá realizar mais coisas sem entrar em depressão.\\

Lembre-se um sorriso dura segundos na realidade, mas vale um dia inteiro. Um grande resultado vale um grande esforço. Não é errado aproveitar a vida. O ser humano busca se afirmar, busca existir para os outros, busca sentido em sua existência, não seja medíocre em sua vontade, mas seja ousado a sua maneira. Não consuma sua vida com sonhos irrealizáveis, crie um caminho realista e objetivo. Não externalize sua incapacidade pela desistência, mas externalize sua determinação com seu esforço. Não se esforce em algo que não funciona, busque soluções pela sua criatividade. Não use métodos adaptados para os outros, adapte o métodos dos outros a sua realidade. Não ignore sua intuição, mas também não abandone a objetividade. Questione, mas se permita explorar caminhos diferentes. Não se importe em fazer valer a pena sua vida se a falta de resultados estiver lhe deixando depressivo, ansioso ou excessivamente cansado, aproveite a vida em seus pequenos detalhes, em seus pequenos momentos e recomponha suas energias, mas não abandone o esforço. Se o mundo estiver lhe fazendo mal, se o mundo estiver tornando turvo sua intuição, dê um tempo para si mesmo, medite, dê um tempo para tudo que lhe faz mal, algumas vezes precisamos escrever em um papel em branco para resolver os problemas. Todos estes conselhos são conselhos de equilíbrio, o antagonismo entre os conselhos deve ser construtivo e não destrutivo, ele deve contribuir para sua existência e não dominar sua consciência. 

%%%%%%%%%%%%%%%%%%%%%%%%%%%%%%%%%%%%%%%%%%%%%%%%%%%%%%%%%%%%%%%%%%%%%%%%%%%%%%%%%%%%%%%%%%%%%%%%%%%%%%%%%%%%
%%%%%%%%%%%%%%%%%%%%%%%%%%%%%%%%%%%%%%%%%%%%%%%%%%%%%%%%%%%%%%%%%%%%%%%%%%%%%%%%%%%%%%%%%%%%%%%%%%%%%%%%%%%%\\
%%%%%%%%%%%%%%%%%%%%%%%%%%%%%%%%%%%%%%%%%%%%%%%%%%%%%%%%%%%%%%%%%%%%%%%%%%%%%%%%%%%%%%%%%%%%%%%%%%%%%%%%%%%%
%%%%%%%%%%%%%%%%%%%%%%%%%%%%%%%%%%%%%%%%%%%%%%%%%%%%%%%%%%%%%%%%%%%%%%%%%%%%%%%%%%%%%%%%%%%%%%%%%%%%%%%%%%%%
%%%%%%%%%%%%%%%%%%%%%%%%%%%%%%%%%%%%%%%%%%%%%%%%%%%%%%%%%%%%%%%%%%%%%%%%%%%%%%%%%%%%%%%%%%%%%%%%%%%%%%%%%%%%
